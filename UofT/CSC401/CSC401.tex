\documentclass{ainote}

%%%%Basic Info%%%%
\author{\ccLogo \,\,Xiyuan Chen}
\title{\textsc{CS401 Natural Language Computing}}
\date{Winter, 2023}
%%%%%%%%%%%%%%%%%%

%%%%Document Beginner%%%%
\begin{document}
\maketitle
\doclicenseThis
\section*{Information}
\begin{itemize}
	\item Course website can be found through \href{https://www.cs.toronto.edu/~raeidsaqur/csc401/}{this link. }
	\item Lectures are fully in-person. But recordings are available.
\end{itemize}
\tableofcontents
\newpage
%%%%%%%%%%%%%%%%%%%%%%%%%

%% Main Body
\section{N-gram Model}

\subsection{Counting}
What are we counting in statistical language models?
\begin{info}
\textbf{\underline{Terminologies:}}
\spskip[1]
\green{\textbf{Token(s)}}: \textbf{instances} of words or punctuation. 每个单词就是一个token
\spskip
\green{\textbf{Type(s)}}: \textbf{kinds} of words or punctuation. 每种单词就是一个type,重复出现的单词只算一次
\spskip
\green{\textbf{Corpus}(pl. corpora)}: A body of language data of a particular sort. 语料库,所有text的集合
\end{info}
We can use the number of tokens or types to statistically model a corpus by computing a \textbf{probability for a sequence of tokens}.
\spskip
\textit{\underline{Attempt 1}}: In a corpus the probability of sequence $P(the\ old\ car) = P(the)P(old)P(car)$. \\
\red{Issues: We ignore the past entirely, the probability of a sequence is the product of \textbf{prior} probabilities.}
\begin{align*}
    P(the\ old\ car)&=P(the)P(old)P(car)\\
    &=P(the)P(car)P(old)\\
    &=P(the\ car\ old)\quad\red{×}
\end{align*}

\textit{\underline{Attempt 2}}: We consider sequences with full dependencies $P(the\ old\ car)=P(the)P(old|the)P(car|the\ old)$.\\
\red{Issue: If we consider \textbf{all} of the past, we will never gather enough data in order to be useful in practice.}\\
For example, the sequence '\textit{the old car}' may never appear in our corpora. 
\begin{align*}
    &P(car|the\ old)=0\\
    \Longrightarrow &P(the\ old\ car)=0\quad\red{×}
\end{align*}
When the sequence gets longer, the conditional probability is very likely to be 0.

\subsection{Definition}
To address the issues mentioned above, we can consider a sequence with fewer dependencies: \textbf{consider only $n$ words(tokens) at a time.}
\begin{info}
    {\large\green{\textbf{N-gram(s)}}}: token sequences of length $N$.
\end{info}
For example, the fragment '\textit{\underline{in this sentence is}}' contains the following \textbf{2-grams}(i.e. bigrams):\\
\textit{(in this), (this sentence), (sentence is)}

\subsubsection{Simple Word prediction}
Given the probabilities of N-grams, we can compute the \textbf{conditional probabilities} of possible subsequent words.
{\small E.g.\begin{align*}
    &P(is\ \text{\blue{\textit{the}}}) > P(is\ \text{\green{\textit{a}}})\\
    \Longrightarrow &P(\text{\blue{\textit{the}}}|is) > P(\text{\green{\textit{a}}}|is)
\end{align*}
Then we would predict '\textit{the last word in this sentence is \textbf{\blue{the}}}'.
}
\spskip[1]
For convenience, we define 
\begin{itemize}
    \item Term count(\green{\textit{\textbf{Count}}}) of term $w$ in corpus $C$ is the number of tokens of term $w$ in $C$. Denoted as $Count(w, C)$.
    \item Relative frequency(\green{\textit{\textbf{$F_C$}}}) is defined relative to the \textbf{total number of tokens} in the corpus, $||C||$. Denoted as $F_C(w)=\frac{Count(w,C)}{||C||}$.
\end{itemize}
\begin{supp}
    In theory, $\displaystyle\lim_{||C||\to \infty}F_C(w)=P(w)$
\end{supp}
Now we can \textbf{approximate} conditional probabilities by counting occurrences in large corpora of data:
\begin{align*}
    P(food|I\ like\ Chinese)&=\frac{P(I\ like\ Chinese\ food)}{P(I\ like\ Chinese)}\\
    &\approx \frac{Count(I\ like\ Chinese\ food)}{Count(I\ like\ Chinese)}
\end{align*}

\subsubsection{Markov Assumption}
\begin{info}
    \textbf{\underline{Markov assumption:}}
    \spskip
    Assume each observation only depends on \textbf{a short linear history} of length $L$. Defined as:
    \begin{align*}
        P(w_n|w_{1:(n-1)})\approx P(w_n|w_{(n-L+1):(n-1)})
    \end{align*}
\end{info}
For example, in \textbf{bigram} version:
\begin{align*}
    P(w_n|w_{1:(n-1)})\approx P(w_n|w_{n-1})
\end{align*}
(把整个sequence的概率简化成最后的$L$个tokens的概率)

\subsubsection{Probability of a Sentence}
\begin{info}
    \textbf{\underline{Probability of a sentence:}}
    \spskip
    The probability of a sentence $s$ is defined as the \textbf{product} of the \textbf{conditional} probabilities of its N-grams:
    \begin{align*}
        P(s)=\prod_{g\in S} P(g) \quad\text{where $g$ is the \textbf{conditional probability} of the N-grams of sentence $s$.}
    \end{align*}
\end{info}
For example, in trigram:
\begin{equation*}
    P(s)=\prod^t_{i=2} P(w_i|w_{i-2}\ w_{i-1})
\end{equation*}

\subsubsection{Performance}
Despite its simplicity, N-gram probabilities can crudely capture \textbf{interesting facts about language and the world}. It can learn:
\begin{itemize}
    \item World Knowledge
    \item Syntax
    \item Discourse(sentence completion)
\end{itemize}

\subsection{Evaluation}
How do we compare two language models?
\subsubsection{Shannon's Method}
\begin{info}
    \textbf{\underline{Shannon's Method:}}
    \begin{itemize}
        \item We can use a language model to \textbf{generate} random sequences.
        \item We want to see sequences that are \textbf{similar} to those we used for \textbf{training}.
    \end{itemize}
\end{info}
\begin{note}
    \textbf{\underline{Shannon's Method with N-grams:}}
    \spskip
    As $N$ \textbf{increases}, the generated text is more \textbf{similar} to the corpus.\\
    $N↑ \Longrightarrow P↓ \Longrightarrow \text{more similar to the dataset}$
    \spskip
    For increasing context, there are fewer possible next words, given the training data.\\
    ($N$变大的时候得到的匹配句子变少,使得生成的句子和这些非常相似。)
\end{note}

\subsubsection{Extrinsic Evaluation}
\underline{Extrinsic evaluation:} in terms of some external measure.\\
The \textbf{utility} of a language model is often determined in \textit{situ}(i.e. in \textbf{practice}).
\spskip
{\small For example, we can
\begin{enumerate}
    \item Alternatively embed LMs $A$ and $B$ into a \textbf{speech recognizer}.
    \item Run speech recognition using each model.
    \item Compare recognition rates between the system that uses LM $A$ and LM $B$.
\end{enumerate}
}
\subsubsection{Intrinsic Evaluation}
\underline{Intrinsic evaluation:} in terms of properties of the LM itself.

\subsubsubsection{Probability of a Corpus}
To measure the intrinsic value of a language model, we first need to estimate the \textbf{probability of a corpus}, $P(C)$.
\begin{info}
    \textbf{\underline{Probability of a Corpus:}}
    \spskip
    For a corpus of sentences, $C$, we make the assumption that the sentences are \textbf{conditionally independent}:
    \begin{equation*}
        P(C)=\prod_i P(s_i)
    \end{equation*}    
\end{info}
\begin{note}
    A language model is better if its corpus probability is higher. (i.e. $P_1(C) > P_2(C) \Longrightarrow P_1\ \text{is better.}$)
\end{note}

\subsubsubsection{Maximum likelihood estimate}
\begin{info}
    \textbf{\underline{Maximum likelihood estimate(MLE)}} of parameters $\theta$ in a model $M$, given training data $T$ is:
    \img{mle.png}{MLE for N-grams}
\end{info}

\subsubsubsection{Perplexity}
\begin{info}
    \textbf{\underline{Perplexity}} on corpus $C$, denoted as $PP(C)$:
    \begin{align*}
        PP(C)&=2^{-\left(\frac{\log_2 P(C)}{||C||}\right)}\\
        &=P(C)^{-1/||C||}
    \end{align*}
\end{info}
Perplexity和entropy有相似之处,系统越混乱(uniform dist)值越高。
\begin{note}
    \begin{itemize}
        \item \textbf{minimizing} perplexity $\equiv$ \textbf{maximizing} probability of corpus.
        \item \textbf{lower} perplexity $\longrightarrow$ \textbf{better} model.
    \end{itemize}
\end{note}

\subsubsection{ZIPF and Natural Distribution in Language}
Problem with N-gram models:
\begin{enumerate}
    \item \textbf{New words} appear often as we read new data. (e.g. Covid-19)
    \item \tdinfo{Too many n-grams in the world!}\textbf{New n-grams} occur even more often.
\end{enumerate}

\subsubsubsection{Sparseness}
\img{sparseness.png}{Sparseness}
\underline{Why does sparseness happen?}
\spskip
The bigram table appears to be filled in \textbf{non-uniformly}. Some words are more \textbf{popular} and will occur in many bigrams just from random chance.

\subsubsubsection{Zipf's Law}
\begin{info}
    \textbf{\underline{Zipf's Law}}:
    \spskip
    Speaker minimizes effort by using a \textbf{small} vocabulary of \textbf{common} words.\\
    Hearer minimizes effort by having a large vocabulary of less ambiguous words.
    \begin{equation*}
        f \propto \frac{1}{r}\quad \text{i.e. for some $k$}\ f\cdot = k
    \end{equation*}
    Frequency and rank are \textbf{inversely} proportional.
\end{info}



\end{document}
