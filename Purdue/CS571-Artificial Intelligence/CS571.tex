\documentclass{ainote}

%%%%Basic Info%%%%
\author{\ccLogo \,\,Xiyuan Chen}
\title{\textsc{CS57100 Artificial Intelligence}}
\date{Fall Term, 2023}
%%%%%%%%%%%%%%%%%%

%%%%Document Beginner%%%%
\begin{document}
\maketitle
\doclicenseThis
\section*{Information}
\begin{itemize}
	\item Syllabus can be found through \href{https://purdue.brightspace.com/d2l/le/content/832225/Home}{this link. }
	\item There will be office hours every weekday. :D
\end{itemize}
\tableofcontents
\newpage
%%%%%%%%%%%%%%%%%%%%%%%%%

%% Main Body
\section{Search}
Solving problems by searching through possible solutions.

\underline{Technique for searching}:
\begin{itemize}
    \item Tree search
    \item Exhaustive vs. heuristic
    \item Dynamic programming
    \item A-star
\end{itemize}

\underline{Problem Types}:
\begin{itemize}
    \item \textbf{Deterministic, fully observable}: Agent knows exactly which state it will be in; solutions in a sequence $\Longrightarrow$ \blue{single-state problem}
    \item \textbf{Non-observable}: Agent may have no idea where it is; solution (if any) is a sequence $\Longrightarrow$ \blue{conformant problem}
    \item \textbf{Nondeterministic and/or partially observable}: percepts provide \brown{new} information about current state, solution is a contingent plan or a policy often \brown{interleave} search, execution $\Longrightarrow$ \blue{contingency problem}
    \item \textbf{Unknown state space}: No info about the problem space. $\Longrightarrow$ \blue{exploration problem("online")}
\end{itemize}

\subsection{Single-state problem}
A single-state problem can be defined by:
\begin{enumerate}
    \item \textbf{Init State}: The starting state of the problem. e.g. the starting location in a route search task.
    \item \textbf{Successor Function}: A set of action-state pairs which contains all states possible in the next action. e.g. $S(Arad)=\{(\text{Arad} \rightarrow \text{Zerind}),(\text{Arad} \rightarrow \text{Gareth})\dots\}$
    \item \textbf{Goal Test}: A function for detecting if the current state is a goal. \hlyellow{There could be multiple goal states!}
    \item \textbf{Path Cost}: (additive), keep track of the total cost of the path.
    \item \textbf{Solution}: A sequence of actions leading from the initial state to a goal state.
\end{enumerate}

\img{1.1-search_example.png}{Example of a single-state problem}

\subsection{Tree search algorithm}
A \textbf{state} is a (representation of) a physical configuration.

A \textbf{node} is a data structure constituting part of a search tree
includes parent, children, depth, path cost $g(x)$.

\hlyellow{States do not have parents, children, depth, or path cost!}

The \textbf{\textit{Expand}} function creates new nodes, filling in the various fields and
using the \textbf{\textit{SuccessorFn}} of the problem to create the corresponding states.
\begin{info}
    Basic idea:
    \begin{enumerate}
        \item Add the initial state node into a fringe.
        \item Get the first node from the fringe. If no node is available in the queue, our search fails. If the node satisfies \textbf{\textit{Goal-Test}}, then return current path as the answer.
        \item Find all possible actions of current node by calling \textbf{\textit{Expand}}. Add resulted nodes into the fringe. Then go back to step 2.
    \end{enumerate}

    The difference between search algorithms is \brown{the order of nodes} we get from the \textbf{\textit{Expand}} function. 
\end{info}
\hlred{$\imp$ Fringe can be any kind of data structure, e.g. Stack, Queue$\dots$}

\img{1.2-algorithm.png}{Tree Search Algorithm}

\subsubsection{Uninformed Search Strategies}
A strategy is defined by picking the \brown{order of node expansion}.

Metrics for evaluating a strategy:
\begin{itemize}
    \item \textbf{completeness}: does it always find a solution if one exists?
    \item \textbf{time complexity}: number of nodes generated/expanded.
    \item \textbf{space complexity}: maximum number of nodes in memory.
    \item \textbf{optimality}: does it always find a least-cost solution?
\end{itemize}

In measuring the time and space complexity, we use:
\begin{itemize}
    \item \blue{$b$} $\longrightarrow$ maximum \textbf{branching factor} of the search tree.
    \item \blue{$d$} $\longrightarrow$ depth of the \textbf{least-cos}t solution.
    \item \blue{$m$} $\longrightarrow$ \textbf{maximum depth} of the state space \hlyellow{(may be $\infty$)}.
\end{itemize}


\subsubsection{Breath First Search(BFS)}
The fringe is a \textbf{FIFO} data structure, e.g. Queue.

Too easy I just won't cover it here.

\underline{Properties}:
\begin{itemize}
    \item Complete? \green{Yes} if $b$ is finite. 
    \item Time? $1+b+b^2+\dots+b^d+b(b^d-1)=\mc{O}(b^{d+1})$

    \hlred{Notice the last term $b(b^d-1)$. BFS expands all the nodes \textbf{in the last level of the tree} regardless that there is no need for}
    \hlred{further expansion, which will cause an exponential increase of space and time compared to the Iterative DFS discussed }
    \hlred{later.}
    \item Space? $\mc{O}(b^{d+1})$, keeps all nodes in memory.
    \item Optimal? Yes (if cost = 1 per step); \textbf{not optimal in general}.
\end{itemize}

\subsubsection{Depth First Search(DFS)}
The fringe is a \textbf{LIFO} data structure, e.g. Stack.

Too easy I just won't cover it here.

\underline{Properties}:
\begin{itemize}
    \item Complete? \red{No}: fails in infinite-depth spaces
    \item Time? $\mc{O}(b^m)$, terrible if $m$ is much larger than $d$.
    \item Space? $\mc{O}(bm)$, linear space !!
    \item Optimal? \red{No}.
\end{itemize}

\subsubsection{Depth Limited Search}
\label{Depth Limited Search}
Limit the depth that original DFS can reach, thus solves the issue for infinite state search.

\img{1.2.4-DLS.png}{Depth Limited Search}

\subsubsection{Iterative Deepening Search}
Iteritively increase the depth limit of \ref{Depth Limited Search} DLS so we can keep exploring until find the goal state while keeping the benifit of DLS.

\img{1.2.5-IDS.png}{Iterative Deepening Search}

\underline{Properties}:
\begin{itemize}
    \item Complete? \green{Yes}
    \item Time? $(d+1)b^0+db^1+(d-1)b^2+\dots+b^d=\mc{O}(b^d)$.
    \item Space? $\mc{O}(bd)$, linear space !!
    \item Optimal? \green{Yes} \textbf{only if} step cost = 1.
\end{itemize}

\begin{info}
    \textbf{Compare IDS with BFS:} IDS does better because \hlyellow{other nodes at depth d are not expanded}.

    e.g. Numerical comparison for b= 10 and d = 5, solution at far right leaf:
    
    $N(\text{IDS})=50 + 400 + 3, 000 + 20, 000 + 100, 000 = 123, 450$

    $N(\text{BFS})=10 + 100 + 1, 000 + 10, 000 + 100, 000 + 999, 990 = 1, 111, 100$
\end{info}

\subsubsection{Uniform-cost Search}
Expand \textbf{least-cost} unexpanded node, e.g. priority queue. Fringe can be a queue ordered by \textbf{path cost}, lowest first.

\img{1.2.5-UCS.png}{Uniform-cost Search}

\underline{Properties}:
\begin{itemize}
    \item Complete? \green{Yes} if cost $\ge c$.
    \item Time? \# of nodes with $g\le$ cost of optimal solution, $\mc{O}(b^{fC^*/cl})$ where $C^*$ is the cost of the optimal solution.
    \item Space? \# of nodes with $g\le$ cost of optimal solution, $\mc{O}(b^{fC^*/cl})$
    \item Optimal? \green{Yes} nodes expanded in increasing order of $g(n)$.
\end{itemize}

\subsection{Comparison}
\begin{center}
    \begin{tabular}{|l|c|c|c|c|c|c|}
    \hline
    Criterion&BFS&UCS&DFS&DLS&IDS&Bidirectional \\
    \hline
    Complete?&Yes&Yes&No&No&Yes&Yes\\
    \hline
    Optinal cost?&Yes&Yes&No&No&Yes&Yes\\
    \hline
    Time&$\mc{O}(b^d)$&$\mc{O}(b^{1+\lfloor C^*/\epsilon\rfloor})$&$\mc{O}(b^m)$&$\mc{O}(b^l)$&$\mc{O}(b^d)$&$\mc{O}(b^{b/2})$\\
    \hline
    Space&$\mc{O}(b^d)$&$\mc{O}(b^{1+\lfloor C^*/\epsilon\rfloor})$&$\mc{O}(bm)$&$\mc{O}(bl)$&$\mc{O}(bd)$&$\mc{O}(b^{b/2})$\\
    \hline
    
    \end{tabular}
\end{center}

\subsection{Greedy Search}
Include a heuristic function $h(n)$ which estimates the cost from $n$ to the closest goal. Greedily expand nodes that have the lowest heuristic cost. 

\underline{Properties}:
\begin{itemize}
    \item Complete? \red{No}, can get stuck in loop.
    \item Time? $\mc{O}(b^m)$ worst case.
    \item Space? $\mc{O}(b^m)$, keeps all nodes in memory.
    \item Optimal? \red{No}
\end{itemize}

\subsection{\hlyellow{$\star$}A* Search}
Idea: avoid expanding paths that are already expensive.

Evaluation function: $f(n)=g(n)+h(n)$, where $g(n)$ is the \textbf{cost so far} and $h(n)$ is the \textbf{heuristic cost}.
\begin{itemize}
    \item \textbf{\hlyellow{\red{Admissible}}}: a heuristic is admissible if $h(n)\le h^*(n)$, where $h^*(n)$ is the \textbf{true} cost from $n$. (Also $h(n)\ge 0$ and $h(G)=0$ for any goal $G$.) i.e. \hlyellow{$h(n)$ never \textbf{overestimates} the actual road distance.}
    
    \hlred{A* search with \textbf{admissible} heuristic is \textbf{optimal}!!}

    \item \textbf{\hlyellow{\red{Consistent}}}: a heuristic is consistent if $h(n)\le c(n,a,n')+h(n')$. i.e. \hlyellow{$f(n)$ is \textbf{nondecreasing} along any path.}
    
    \hlred{\textbf{Consistency} is stronger than \textbf{admissible}}.

    Prove:
    \begin{align*}
        f(n')&=g(n')+h(n') \\
        &=g(n)+c(n,a,n')+g(n') \\
        &\ge g(n)+h(n) \\
        &=f(n)
    \end{align*}
\end{itemize} 

\begin{info}
    Prove the optimality of A* if \textbf{admissible}:

    Suppose some suboptimal goal $G_2$ has been generated and is in the queue. Let $n$ be an unexpanded node on a shortest path to an optimal goal $G_1$.
    \begin{align*}
        f(G_2)&=g(G_2) \qquad\text{since $h(G_2)=0$} \\
        &>g(G_1) \qquad\text{since $G_2$ is suboptimal, not optimal} \\
        &\ge f(n) \qquad\text{since $h$ is admissible}
    \end{align*}
    Since $f(G_2)>f(n)$, $A^*$, by definition, will \textbf{never} select $G_2$ for expansion.

    \textbf{Lemma:} $A^∗$ expands nodes in order of \hlyellow{increasing $f$ value}.
\end{info}

\end{document}