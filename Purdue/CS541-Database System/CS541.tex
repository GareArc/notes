\documentclass{ainote}

%%%%Basic Info%%%%
\author{\ccLogo \,\,Xiyuan Chen}
\title{\textsc{CS54100 
Database Systems}}
\date{Fall Term, 2023}
%%%%%%%%%%%%%%%%%%

%%%%Document Beginner%%%%
\begin{document}
\maketitle
\doclicenseThis
\section*{Information}
\begin{itemize}
        \item Instructor: \href{https://www.cs.purdue.edu/homes/aref/}{Walid G Aref}
	\item Syllabus can be found through \href{https://purdue.brightspace.com/d2l/le/content/832018/Home}{this link. }
	\item Group project is required. 
\end{itemize}
\tableofcontents
\newpage
%%%%%%%%%%%%%%%%%%%%%%%%%

%% Main Body
\section{Introduction}
\subsection{Focus of the course}
In this course we are focusing on the DBMS Engine instead of using DBMS as a tool to build applications.
\img{1.1-db_sys.png}{Database System Overview}
\begin{info}
    $\text{Database System}=\text{Application Program}+\text{DBMS}+\text{Database}$
\end{info}

\subsection{Course description}
The course is organized along \green{three} main dimensions:
\img{1.2-dimensions.png}{Three Main Dimensions}
\begin{enumerate}
    \item \textbf{Hardware}: Match new types of storage technologies, new CPU architectures etc.
    \item \textbf{Relational}: How the \green{\textbf{Relational Database systems}} work. e.g. the one-size-fits-all philosophy.
    \item \textbf{Beyond}: Customized and tailored systems for certain classes of applications and workloads. e.g. NoSQL System \& Big Data System.
\end{enumerate}

\subsection{Course Topics}
The topics covered in this course include:
\begin{itemize}
    \item DBMS Concepts and Architecture.
    \item \tdinfo{This is just a quick review.}{Overview of Relational Databases:} Relational Algebra, and Calculus, Modeling, Normalization, Tuning, and Security.
    \item Storage Technologies: Volatile and Non-volatile.
    \item Data Organization: Row and Column Stores.
    \item Single and Multi-dimensional Indexing.
    \item Extensible indexing: Generalized Search Trees.
    \item Write-optimized Indexing Techniques.
    \item Query Evaluation and Optimization.
    \item NoSQL, Multi-model Databases, and Poly-Stores.
    \item Transaction Management, Concurrency Control, and Recovery.
    \item Big Data Streaming Systems: Concepts and Trends.
    \item Main-memory Database Techniques.
\end{itemize}

\section{Data Models}
\subsection{Evolution and the Relational models}
\subsubsection{File systems vs. DBMS}
Reasons for having DBMS:
\begin{itemize}
    \item Data independence: no need to change programs when the underlying data orgnizaton changes.
    \item Data Integrity: data is consistent across the data files.
    \item Concurrent access: avoid race and other inconsistencies.
    \item Support transaction: make transaction as an atomic semantic unit.
    \item Crash recovery: surviving crashes and data durability.
    \item Security and access control: permissions and administration settings.
\end{itemize}

\subsubsection{Relational data model}
\blue{\textbf{\textit{Relation}}}:
\begin{itemize}
    \item a table with rows and columns.
    \item has \textbf{schema} that describes the data stored in the columns.
\end{itemize}
\blue{\textbf{\textit{Constraints}}}: guarantee the integrity of the related data.

\blue{\textbf{\textit{Relational Database}}}: a set of inter-related tables (relations). Each relation contains two parts:
\begin{enumerate}
    \item \textbf{\textit{Schema}}: specifies
        \begin{itemize}
            \item the relation name
            \item names and types of all attributes
            \item the number of columns(attributes) is called \brown{\textbf{arity}} or \brown{\textbf{degree}}.
            \item \brown{\textbf{type}} determines the set of acceptable values for the attribute.
        \end{itemize}
    \item \textbf{\textit{Instance}}: the actual data that fills the schema.
        \begin{itemize}
            \item it is a \textbf{set} of tuples, i.e. no order, no duplicates.
            \item the number of rows(tuples) is called \brown{\textbf{cardinality}}.
        \end{itemize}
\end{enumerate}

\blue{\textbf{\textit{Keys}}}:
a key is a subset of the attributes that uniquely identifies a tuple in the relation.
\begin{enumerate}
    \item \textbf{\textit{Primary key}}: Is the candidate key that we choose to be "the key" for a given table.
    \item \textbf{\textit{Unique key}}: Are the other candidate keys for a given table that are not selected to be the primary key.
    \item \textbf{\textit{Candidate key}}: \tdinfo{the minimum key set.}{the minimal subset of attributes that uniquely identify a tuple in the table.}
    \item \textbf{\textit{Super key}}: \tdinfo{other than candidate key, just like superset.}{A primary or unique key that is augmented by additional attributes.}
\end{enumerate}
\begin{note}
    \underline{\textbf{Who determines the keys?}}
    
    Keys are determined by the \textbf{system designer} or \textbf{system analyst} depending on the real world scenario.
\end{note}

\blue{\textbf{\textit{Foreign keys}}}: An attribute $B_1$ in table $B$ that is the \textbf{key}(primary or unique key) $A_1$ of another table $A$.
\begin{itemize}
    \item Values of attribute $B_1$ is restricted to those in table $A_1$.
    \item $B_1$ is termed \textbf{Foreign key} to $A$.
\end{itemize}

\subsubsection{Query languages for the relational model}
There are two categories of query language for relational model:
\begin{enumerate}
    \item \textbf{Imperative(Procedural)}: 
        \begin{itemize}
            \item Specifies the detailed steps to be taken to evaluate a user's query.
            \item e.g. Relational Algebra
        \end{itemize}
    \item \textbf{Declarative (Non-procedural)}:
        \begin{itemize}
            \item Specifies the query that the user needs answered and \textbf{not} how to answer it.
            \item e.g. Relational Calculus(SQL, QBE).
        \end{itemize}
\end{enumerate}

\subsection{The design process}
When we design our database for a specific application, we need to follow a design process.
\img{2.2-design_process.png}{Design Process}
\begin{enumerate}
    \item \textbf{Entity Relational model}(ER): is a general modeling tool. It helps us map a real-world scenario into an ER-model-based design.
    
    \hlred{The ER model is \textbf{not} necessarily tied to the actual relational DBMS. It is just a conceptual model.}
    \item Translate ER model to your corresponding model that is adopted by your underlying DBMS.
        \begin{itemize}
            \item If use NoSQL Graph Data System $\Longrightarrow$ Graph Data Model
            \item If use Relational DBMS $\Longrightarrow$ Relational Model
        \end{itemize}
\end{enumerate}
\begin{info}
    \textbf{Differences Among The Target Models}

    The \textbf{relational model} is \hlyellow{\textbf{Schema-first}} since we define the schema before inserting data. $\Longrightarrow$ more efficient

    The \textbf{NoSQL models} are \hlyellow{\textbf{Schema-last}} since we can modify the data structure freely. $\Longrightarrow$ more flexible
\end{info}
\subsubsection{The road map}
\begin{enumerate}
    \item Domain Experts

    Tell their story - What they need
    \item The System Analyst

    Use some modeling language to map the domain expert needs to an intermediate model that the database developers can use to design their own system-dependent database application.
    e.g. UML, ER Model

    Tasks:
    \begin{itemize}
        \item Transferring the design document to model
        \item Eliminating redundancy
    \end{itemize}
\end{enumerate}

\subsection{ER Model}
\begin{supp}
    There are lots of images and graphs involved so please check the \href{https://purdue.brightspace.com/d2l/le/content/832018/viewContent/14053416/View}{lecture slides} for details.
\end{supp}

We use ER model for analyzing and modeling the real-world scenario.

Elements in ER model:
\begin{itemize}
    \item \textbf{Entity}: real-world objects that is of interest in the scope of this scenario. \green{Represented by a rectangle.}
    \item \textbf{Attributes}: belongs to entities. \green{Represented by a oval}
    \item \hlyellow{$\star$} \textbf{Entity Set}: \tdinfo{It's just the entity but consider in the way of instances.}{A group or collection of entities that share the same attributes.} e.g. a set of instructors. Each entity set has one or more attributes that serves as the \brown{\textbf{key}}, which is \green{represented as a underline} inside the oval. 
    \item \textbf{Relationship}: associate two or more entities. It \textbf{can} have its own attributes. \green{Represented by a diamond}.
    \item \textbf{Relationship Set}: similar to Entity Set.
    \begin{itemize}
        \item An entity set can participate in multiple relationship sets.
        \item An entity set can participate in the \textbf{same} relationship but as different \textbf{roles}. e.g. employees can manage or be managed by other employees.
        \item \textbf{N-nary Relationship} refers to the number of entities involved in a relationship. (at least binary even for the self-pointing example because we count the number of arms, not entity sets.)
    \end{itemize}
    \item \textbf{Cardinality Constraints}: 
    \img{2.3-cardinality.png}{Relationship Mapping}
    \begin{itemize}
        \item \tdinfo{can have 0 arm as well}{1-to-1: every entity can have \textbf{at most} one arm reaching out.}\green{Represented by arrow line on other side.}
        \item 1-to-Many: at most one arm for left and no limit for the right.
        \item Many-to-1: reversed case of 1-to-Many.
        \item Many-to-Many: no limit at all.
    \end{itemize}
    \item \textbf{Total participation}: all instances in the entity set must have at least one arm. (Can combine with 1-to-1 to disallow no-connection item.) \green{Represented by bold line.}
    \item \textbf{LowerLimit-UpperLimit}: A more complex way to manage cardinality constraint. Basically we write out explicitly how many arms are allowed to reach out from each entity in the relationship.
    \img{2.3-cardlimit.png}{LowerLimit-UpperLimit Example}

    \item \textbf{Weak Entity}: \tdinfo{like an external \textbf{attribute entity }that is attached to an \textbf{owner entity} by a relationship}{one that can be identified uniquely \textbf{only} by considering the primary key of another \textbf{owner entity}.} \hlyellow{Weak entity must have \textbf{total participation}!!!} \green{Represented by bold border.}
    \img{2.3-weakentity.png}{Weak Entity Example}

    \item \textbf{Specializations and Generalizations}: \tdinfo{like \textbf{parent class} and \textbf{child class} in OOP}{Identify a subset within an entity set that have some commonalities and that are distinguishable from other entities in the same entity set.}
    \begin{itemize}
        \item Overlap Constraint: can one entity belong to multiple specializations? (can instance $a$ belongs to many child classes?)
        \item Covering Constraints: Can one entity in the superset but not in any of the subsets? (Is the parent class abstract?)
    \end{itemize}

    \item \textbf{Aggregation}: when we want to create a relationship that relates an entity to other relationships and entities. Treat a relationship set as an entity set and associate it with another relationship. The new aggregated entity set is not an actual table in database, but created in runtime(like a pseudo table).

    \item \textbf{Advanced Attributes}: Attributes may also allow
    \begin{itemize}
        \item Composite: be a hierarchy of any depth and contains component attributes.
        \item Multi-valued: a list of values.
        \item Derived: computed from other attributes.
    \end{itemize}
\end{itemize}

\end{document}